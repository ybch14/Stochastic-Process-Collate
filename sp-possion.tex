\setcounter{section}{0}
\setcounter{subsection}{0}

\chapter{泊松过程}

\section{计算相关函数和功率谱密度}

\subsection{\hyperref[A2014-10]{【2014-10】}}\label{Q2014-10}

设 $N(t)$ 为泊松过程,参数为 $\lambda$。随机变量 $A\sim N(0, \sigma^2)$。考虑随机过程 $X(t)$,
$$
X(t)=A\cos(2\pi ft+\pi N(t))
$$
计算 $X(t)$ 的均值和相关系数。

\subsection{\hyperref[A2007-3]{【2007-3】}}\label{Q2007-3}

考虑随机过程 $X(t)=Z^{N(t)}$,其中 $N(t)$ 是标准泊松过程,参数为 $\lambda$,$P(Z=1)=p,\ P(Z=-1)=1-p$,且假定 $N(t)$ 和 $Z$ 统计独立,请计算 $X(t)$ 的自相关系数。

\subsection{\hyperref[A2007-4]{【2007-4】}}\label{Q2007-4}

设随机过程 $Y(t)=X_{N(t)}$,其中$N(t)$ 是标准泊松过程,参数为 $\lambda$,$\{X_n\}$ 为独立同分布随机变量,均值为 $m$,方差为 $\sigma^2$,请计算 $Y(t)$ 的功率谱密度。
\\\\
提示:第 1、2 题令 $N(t)=k$,分奇偶性讨论;第 3 题 $R_Y=E[X_N(t_1)X_N(t_2)]$,分 $N(t_1)$ 和 $N(t_2)$ 是否相等讨论。
\\\\
\section{泊松过程的和}

\subsection{\hyperref[A2010-3]{【2010-3】}男女生到校}\label{Q2010-3}

假定学校早上 7 点开门,男生按照强度为 $\lambda$ 的泊松流到达学校,女生按照
强度为 $\mu$ 的泊松流到达学校,男女生的到达行为相互独立。试计算从 7 点开始算起,到达学校的头两个学生性别不同的概率.

\subsection{\hyperref[A2010-5]{【2010-5】}两个服务台}\label{Q2010-5}

假定某银行有两个服务台,张先生到达银行的时候,两个服务台都被顾客占用,且没有顾客在等待。设两个服务台的服务时间分别服从参数为 $\lambda_1$ 和 $\lambda_2$ 的指数分布,且不同顾客间的服务时间相互独立,试计算张先生成为三个人当中离开银行最晚的人的概率。

\subsection{\hyperref[A2008-1]{【2008-1】}N 个服务台}\label{Q2008-1}

假定银行有 N 个服务台,各个服务台为一个顾客服务所需要的时间是独立同指数分布的随机变量,参数为 $\lambda$。有 N+l 个顾客同时到达了银行,其中一个顾客为了攒“人品”,主动发扬风格,让其他 N 个人先接受服务,自己等待;当先接受服务的人中有人服务结東离开后.该好心人开始接受服务。试计算,该好心人成为 N+l 个人中最后一个完成服务离开的人的概率。
\\\\
提示:第 1、2 题:本质为计算条件概率。第 3 题:直观上,由于指数分布具有无记忆性,答案为 1/N;本质上,考查顺序统计量,需计算 $P(T_{N+1}\le \max(T_1,\dots,T_N)- \min(T_1,\dots,T_N))$。

\section{顺序统计量}

\subsection{\hyperref[A2009-6]{【2009-6】}}\label{Q2009-6}

设三台机器组成串行系统,任何一台机器停止工作都会使得系统失效,但是
要等到三台机器都停止工作后,系统才开始进行维护修理。假设各台机器的
无故障持续工作时间服从参数为 $\lambda$ 的指数分布,且相互独立。设 0 时刻为起
始时刻,三台机器同时启动开始正常工作。试计算:在某一指定时刻,发
现系统己经失效的条件下,系统开始进行维护修理的时刻 $T$ 的分布函数。

\subsection{\hyperref[A2008-2]{【2008-2】}}\label{Q2008-2}

考虑泊松过程 $N(t)$,强度为 $\lambda$。设事件间隔为 $T_1, T_2, \dots$,令
$$
M=\min\{n|T_n=\max(T_1, \dots, T_n)\}
$$
试计算$E(T_1+\dots+T_M)$。

\subsection{\hyperref[A2014-1]{【2014-1】}}\label{Q2014-1}

到达公交汽车站的公交车服从参数为 $\lambda$ 的泊松过程。某乘客到达公交汽车站,记 A 为自上
一趟公交车到站时间起,直至当前时刻所经历的时间,B 为自当前时刻起,直到下一趟公交车
到站所需的时间。计算$E(\min(A, B))$。

\subsection{【2009-4】等待时间的和}

设某起始车站有快、慢两种车,快车开车的间隔为参数为 3 的指数分布,慢车开车的间隔为参数为 10 的指数分布,到达该车站的乘客服从参数为 1 的泊松流,且一旦来车,乘客无论车的快慢,全部上车。设快车从起始站到终点站的运行时间为 $T_1$,慢车为 $T_2$,$T_1$ 和 $T_2$ 均为确定常数,且所有顾客均以终点站为目的地。试问:$T_1$ 和 $T_2$ 的差为多大时.才能够使得乘坐快车的乘客的平均花费时间之和小于秉坐慢车的乘客的平均花费时间之和。这里花费时间包括等车时间和运行时间。

\section{其他}

\subsection{\hyperref[A2009-3]{【2009-3】}分奇偶讨论}\label{Q2009-3}

某台机器在运转,无故障持续时间服从参数为 $\lambda$ 的指数分布:如果出现故障,即刻由修理工进行修理,修理时间也服从参数为 $\lambda$ 的指数分布:修理好之后即刻重新开始运转,如此循环往复。设各段无故障工作时间与各段修理时间均独立,修理工每修好一台机器得到报酬 1 元,试计算 $[0, t]$ 内该修理工所得报酬的均值(这里假定机器从 0 时刻开始运转)。

\subsection{\hyperref[A2007-8]{【2007-8】}}\label{Q2007-8}

考虑泊松过程 $N(t)$,令 $T_n$ 为第 $n$ 次事件发生时刻,给定时刻 t,设
$$
A(t)=t-T_{N(t)},\ B(t)=T_{N(t)+1}-t
$$
请计算 $A(t)$ 和 $B(t)$ 的分布及联合分布,判断这两者是否独立。然后请计算 $A(t)+B(t)$
的均值。

\subsection{\hyperref[A2014-6]{【2014-6】}}\label{Q2014-6}

设跑步者 A,B,C 在操场上跑,每圈所用时间分别服从相互独立的指数分布,参数分别为
$\lambda_A=21,\ \lambda_B=23,\ \lambda_C=24$。在每次完成一圈时,跑步者会喝一杯或两杯水,喝一杯水和两杯
水的概率分别为 1/3 和 2/3 ,且跑步者每次喝水之间是相互独立的。假设喝水时间可以忽略。

(1)试求在第一个小时内三位跑步者消耗的总水量的平均值(以杯为单位)。

(2)试求 A 在 B,C 之前完成第一圈的概率。

(3)己知A己经跑了1/4小时,且 A 正在第二圈。试求 A 第二圈所花时间的平均值。